\NeedsTeXFormat{LaTeX2e}
\documentclass[10pt]{amsart}
\usepackage{subcaption}
\usepackage{amsmath,amssymb,verbatim,amsthm}
\usepackage[pdftex]{graphicx}
\usepackage{wrapfig}
\usepackage[labelfont=bf]{caption}
\usepackage{listings, fancyref}
\usepackage{fullpage, nicefrac}

\usepackage{hyperref}

\begin{document}

\title{Documentation to Accompany Materials Fingerprinting Manuscript}
\author{Adam Spannaus$^1$}
\thanks{This manuscript has been authored by UT-Battelle, LLC under Contract No. DE-AC05-00OR22725 
	with the U.S. Department of Energy. The United States Government retains and the publisher, 
	by accepting the article for publication, acknowledges that the United States Government 
	retains a non-exclusive, paid-up, irrevocable,world-wide license to publish or reproduce 
	the published form of this manuscript, or allow others to do so, for United States 
	Government purposes. The Department of Energy will provide public access to these 
	results of federally sponsored research in accordance with the DOE Public 
	Access Plan (http://energy.gov/downloads/doe-public-access-plan).}
\address[1]{Oak Ridge National Laboratory, Oak Ridge, TN 37830}
\author{Kody J. H. Law$^2$}
\address[2]{School of Mathematics, University of Manchester, Manchester, UK}
\author{Piotr Luszczek$^3$}
\address[3]{Innovative Computing Laboratory, University of Tennessee, Knoxville, TN 37996}
\author{Farzana Nasrin$^4$}
\address[4]{Department of Mathematics, University of Hawaii at Manoa, Honolulu, HI 96822}
\author{Cassie Putman Micucci$^5$}
\address[5]{Eastman Chemical Company, Kingsport, TN 37662}
\author{Peter K. Liaw$^6$}
\address[6]{Department of Materials Science and Engineering, University of Tennessee, Knoxville, TN 37996}
\author{Louis J. Santodonato$^7$}
\address[7]{Advanced Research Systems, Inc., Macungie, PA 18062}
\author{David  J. Keffer$^6,*$}
\address[6]{Department of Materials Science and Engineering, University of Tennessee, Knoxville, TN 37996}
\email[Corresponding author]{dkeffer@utk.edu}
\author{Vasileios Maroulas$^{8,*}$}
\address[8]{Department of Mathematics, University of Tennessee, Knoxville, TN 37996}
\email[Corresponding author]{vmaroula@utk.edu}
\maketitle


This document provides descriptions of the python source files and 
data to accompany the Materials Fingerprinting manuscript of Spannaus et al.
This software depends on the following libraries and has been tested with the versions listed:
\begin{itemize}
	\item python - 3.8,
	\item numpy - 1.18 and 1.19,
	\item sklearn - 0.23,
	\item scipy - 1.4,
	\item matplotlib - 3.2, and 3.3,
	\item \href{https://ripser.scikit-tda.org/}{ripser} - 0.4.1 and 0.5.2.
\end{itemize}

The software has been tested with the listed versions of these libraries on linux and Mac OSX.
To install any missing dependencies, all may be installed via `pip', eg,
\texttt{\$ pip install Ripser}, from the command line.

To get started after installing the required libraries, download
the repository and save the directory to the Desktop. All 
commands are to be run from the terminal in the Materials-Fingerprinting
directory. The program is set to compute the distances in parallel and
cannot be run from within an \texttt{ipython} session. 
To run the binary classification, type \texttt{\$ python tda\_classify2.py},
and the three-way classification, enter \texttt{\$ python tda\_classify3.py},
from the command line in the fingerprinting directory, which will run the 
classification with the default
settings of added noise ($\sigma=0.25$) and percent missing (33\%). 
Furthermore the code is set to
run the classification method on a subset of the data presented in the 
manuscript. The number of structures, along with the added noise and sparsity may be changed
in the code as described below.


We give a description of each file.
\begin{itemize}
	\item \texttt{tda\_classify2.py:} 
		These files run the main classification routines for binary scenario. 
		The default setting is for 
		an even split between 500 BCC and FCC structures.
		If you want to investigate a different proportion, that value
		may be set on line 181, and the number of structures is set on line 104.
		To change the amount of noise or percent missing, you may specify
		these values in line 179 and 180 in \texttt{tda\_classify2.py}.
		
	\item \texttt{tda\_classify3.py:}
		These files run the main classification routines for the
		multi-class classification setting. A multi-class scenario, the number of BCC and FCC structures
		to use may be set on line 104 of the file \texttt{tda\_classify3.py} and 
		the number of HCP structures may be set on line 105 of the same
		file. If you want to investigate a different proportion, that value
		may be set on line 189.
		To change the amount of noise or percent missing, you may specify
		these values in line 187 and 188 in the file \texttt{tda\_classify3.py}.
		
	\item \texttt{classify\_utils.py:}
		This file contains the functions to load the data and create 
		persistence diagrams. The path to load the data is presently set to use 
			the Desktop folder. If the code is installed in a different directory,
			from the Desktop, the path must be specified on line 114. If this path
			is incorrect, the program will prompt for the correct path to the data.
			
		\item \texttt{dist.py:}
			This file contains the functions and classes 
			for constructing the feature matrix for the classification 
			algorithm. It is set to process all the distances in parallel, and 
			must be run from the command line. It will not work correctly from an 
			interactive python session, such as \texttt{ipython} within Spyder.
			
		\item \texttt{distances.py:}
			Contains the $d_p^c$, equation 3.1 in the manscript, and Wasserstein persistence diagram 
			distance function definitions.
\end{itemize}


\end{document}
